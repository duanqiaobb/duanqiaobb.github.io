% Created 2017-04-27 Thu 15:45
% Intended LaTeX compiler: pdflatex
\documentclass[presentation]{beamer}
\usepackage[utf8]{inputenc}
\usepackage[T1]{fontenc}
\usepackage{graphicx}
\usepackage{grffile}
\usepackage{longtable}
\usepackage{wrapfig}
\usepackage{rotating}
\usepackage[normalem]{ulem}
\usepackage{amsmath}
\usepackage{textcomp}
\usepackage{amssymb}
\usepackage{capt-of}
\usepackage{hyperref}
\usetheme{default}
\date{\today}
\title{}
\hypersetup{
 pdfauthor={},
 pdftitle={},
 pdfkeywords={},
 pdfsubject={},
 pdfcreator={Emacs 25.2.1 (Org mode 9.0.5)}, 
 pdflang={English}}
\begin{document}

\begin{frame}{Outline}
\tableofcontents
\end{frame}

\begin{frame}[label={sec:org4adfcd3}]{}
\begin{itemize}
\item 30前将普通傅里叶变化的算法和web程序写出来以及论文弄好(还有4天)
\item 有道口语一课(切记)
\end{itemize}
\end{frame}


\begin{frame}[fragile,label={sec:org77cb714}]{}
 \begin{block}{C语言中的复数操作}
 复数在数学运算中十分重要,在编写数值运算或者算法的时候,我们会用到复数这种概念。
那么,复数在C/C++语言中是如何表示的呢?我们接下来一一介绍。

\begin{itemize}
\item C语言中复数
\end{itemize}

 在数学中一个复数可以定义为 \(z=a + bi\) 的形式。 \texttt{C} 语言在 \texttt{ISO C99} 时就引入了复数类型。它是通过 \texttt{complex.h} 中定义的。
我们可以使用 \texttt{complex} , \texttt{\_\_complex\_\_} , 或 \texttt{\_ComplexI} 类型符号来表示。在C语言中有三种复数类型,分别为 \texttt{float complex} , \texttt{double complex} , \texttt{long double complex} 。他们之间
的区别就是表示复数中实部和虚步的数的数据类型不同。 \texttt{complex} 其实就是一个数组,数组中有两个元素,一个表示复数的实部,一个表示复数的虚部。

\begin{itemize}
\item 定义一个复数
\end{itemize}

 在 \texttt{complex.h} 头文件中定义了两个宏 \texttt{\_Complex\_I} 和 \texttt{I} 来创建一个复数。

\begin{verbatim}
Macro: const float complex _Complex_I;
Macro: const float complex  I;
\end{verbatim}



 这两个宏表示复数 \(0+1i\) , 我们可是使用这个单位复数来创建任何复数。

\begin{verbatim}
#include <stdio.h>
#include <complex.h>

int main(int argc, char *argv[])
{
  complex  double  a = 3.0 + 4.0 * _Complex_I;
  __complex__ double b = 4.0 + 5.0 * _Complex_I;
  _Complex  double c = 5.0 + 6.0 * _Complex_I;

  printf("a=%f+%fi\n", creal(a),cimag(a));
  printf("b=%f+%fi\n", creal(b), cimag(b));
  printf("c=%f+%fi\n", creal(c), cimag(c));



  printf("the arg of a is %d", carg(a));

  return 0;
}
\end{verbatim}

\begin{verbatim}
a=3.000000+4.000000i
b=4.000000+5.000000i
c=5.000000+6.000000i
the arg of a is 176
\end{verbatim}

\begin{itemize}
\item 复数的基本操作函数
\end{itemize}

 在 \texttt{complex.h} 头文件中定义一些对复数的基本操作的函数。


\begin{center}
\begin{tabular}{ll}
函数 & 功能\\
\hline
creal & 获取复数的实部\\
cimag & 获取复数的虚部\\
conj & 获取复数的共轭\\
carg & 代表,复平面上穿过原点和复数在复平面表示的点,的直线和实数轴之间的夹角\\
cproj & 返回复数在黎曼球面上的投影\\
\end{tabular}
\end{center}


\begin{itemize}
\item 复数的数值操作
\end{itemize}

 复数类型也可以像普通数值类型, \texttt{int, double, long} 一样进行直接使用数值操作符号,进行数值操作。

\begin{verbatim}
#include <stdio.h>
#include <complex.h>

int main(int argc, char *argv[])
{
  complex  double  a = 3.0 + 4.0 * _Complex_I;
  __complex__ double b = 4.0 + 5.0 * _Complex_I;
  _Complex  double c = 5.0 + 6.0 * _Complex_I;

  complex double d =a + b;
  complex double f = a *b ;
  complex double g = a/b;

  printf ("d=a+b=%f+%fi\n",creal(d),cimag(d));
  printf ("f=a*b=%f+%fi\n",creal(f),cimag(f));
  printf("g=a/b=%f+%fi\n",creal(g),cimag(g));

  return 0;
}
\end{verbatim}

\begin{verbatim}
d=a+b=7.000000+9.000000i
f=a*b=-8.000000+31.000000i
g=a/b=0.780488+0.024390i
\end{verbatim}
\end{block}

\begin{block}{C++中的复数}
 C++中的复数操作在C语言基础上引进了面向对象的特性,在 \texttt{C++} 头文件在 \texttt{complex} 中定义了一个 \texttt{complex} 类用来表示复数。同时为了兼容 \texttt{C} 的 \texttt{complex} 类型,也提供了一个 \texttt{complex.h} 的头文件。
不同的是,在 \texttt{complex.h} 头文件中, \texttt{complex} 关键字被废弃,只能使用 \texttt{\_Complex} 和 \texttt{\_\_complex\_\_} 来使用 C风格的复数形式。

\begin{itemize}
\item 复数的定义
\end{itemize}

 在C++中可以使用两种方式定义一个复数,一个使用C风格 \texttt{\_Complex} 和 \texttt{\_\_complex\_\_} ,一个是使用 \texttt{complex} 类。


\begin{verbatim}
#include <iostream>
#include <complex>
#include <complex.h>
using namespace std;
int main(int argc, char *argv[])
{
  complex<double> mycomplex(2.000,2);
  _Complex double  mycomplex2 = 2.000 + 3I;
  __complex__ double mycomplex3 = 2.000 + 4I;

  cout << mycomplex << endl;
  cout << "(" << creal(mycomplex2) << "," << cimag(mycomplex2) << ")" << endl;
  cout << "(" << creal(mycomplex3) << "," << cimag(mycomplex3) << ")" << endl;
  return 0;
}
\end{verbatim}

\begin{verbatim}
(2,2)
(2,3)
(2,4)
\end{verbatim}


\begin{itemize}
\item 复数的基本操作函数
\end{itemize}

 在C++中既可以使用 \texttt{C} 风格的相关函数。


\begin{center}
\begin{tabular}{ll}
函数 & 功能\\
\hline
creal & 获取复数的实部\\
cimag & 获取复数的虚部\\
conj & 获取复数的共轭\\
carg & 获取,复平面上穿过原点和复数在复平面表示的点,的直线和实数轴之间的夹角\\
cproj & 返回复数在黎曼球面上的投影\\
\end{tabular}
\end{center}

 也可以使用 \texttt{complex} 类中的相关操作方法。


\begin{center}
\begin{tabular}{ll}
方法 & 功能\\
\hline
complex.real() & 获取复数的实部\\
complex.imag() & 获取复数的虚部\\
complex.abs() & 获取复数的绝对值\\
complex.arg() & 获取,复平面上穿过原点和复数在复平面表示的点,的直线和实数轴之间的夹角\\
complex.norm() & 获取复数的范数\\
complex.conj() & 获取复数的共轭\\
complex.polar() & 获取极坐标对应的复数\\
complex.proj() & 返回复数在黎曼球面上的投影\\
\end{tabular}
\end{center}
\end{block}
\end{frame}




\begin{frame}[label={sec:org2e63169}]{}
\begin{itemize}
\item 详细了解cpp中的function->return type的用法
\item 写一个管理markdown的博客vim插件(参考vim的vimwiki插件)
\item 了解C++中的异常的使用
\item 在emacs或者vim中实现一个小的智能程序
\item 写一个类似Mac中的高亮当前鼠标位置的效果(Linux下)
\item 在vim移植emacs的org-mode(针对markdown)
\item 命令行的qq聊天(vim-plugin,emacs-plugin)
\item 探究真正意义上的人工智能
\item 学习黑客的思维
\item 写个备份Gentoo Linux的脚本(Live版本,iso格式)
\item 写个像Grammerly的功能的软件
\item 写方面绑定手机的各种账号更换
\item 看王垠的40行代码
\item 学习scheme语言
\item 详细看《Chapter 13. Copy contral》
\item 详细看《 Chapter 12. Dynamic Memory》
\item 详细看《13.1.4(p. 504)》
\item 了解C++里面的lamda表达式
\item C++的线程相关操作
\item 看825页的dynamic\(_{\text{cast}}\)(看完虚函数再看)
\item 《 CPP prime 》 看到Circumventing the Virtual Mechanism
\item grub,UEFI,Legacy,Windows Loader,以及其他相关加载器的相关知识
\end{itemize}
\end{frame}

\begin{frame}[label={sec:org93c62bc}]{}
\end{frame}
\end{document}
